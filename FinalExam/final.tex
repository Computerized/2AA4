\documentclass[12pt,fleqn]{examtst}
\usepackage{graphicx}
\usepackage{amssymb}
\usepackage{amsmath}
\usepackage{listings}
\usepackage{multirow}
\usepackage{multicol}
\usepackage{hhline}
\usepackage{booktabs}
\usepackage{url}
\usepackage{enumerate}
\usepackage{hyperref}
%% Comments

\usepackage{color}

\newif\ifcomments\commentstrue

\ifcomments
\newcommand{\authornote}[3]{\textcolor{#1}{[#3 ---#2]}}
\newcommand{\todo}[1]{\textcolor{red}{[TODO: #1]}}
\else
\newcommand{\authornote}[3]{}
\newcommand{\todo}[1]{}
\fi

\newcommand{\wss}[1]{\authornote{blue}{SS}{#1}}

\begin{document}

\newcommand{\soln}{n} %y for yes and n for no

\lstset{language=python, basicstyle=\ttfamily, breaklines=true,
  showspaces=false, showstringspaces=false, breakatwhitespace=true, texcl=true,
  escapeinside={\%*}{*)}}

\newcommand{\codeit}[1]{\texttt{\textit{#1}}}

\begin{center}
  {\large \bf COMP SCI 2ME3 and SFWR ENG 2AA4 Final Examination}\\[1ex]
  {\large \bf McMaster University}\\[1ex]
  \ifthenelse{\equal{\soln}{y}}{\large {\bf Answer Key:} Large arrow
    ($\Longleftarrow$) for correct% , small ($\leftarrow$) for partially
    % correct
  }{}
\end{center}

\medskip

\noindent
DAY CLASS, \textbf{Version 1}  \hfill Dr.~S.~Smith \\
DURATION OF EXAMINATION: 2.5 hours (+ 30 minutes buffer time)\\
MCMASTER UNIVERSITY FINAL EXAMINATION \hfill April 28, 2021

\medskip

\noindent
\rule[3 mm]{\textwidth}{0.5mm}

%\begin{minipage}[t]{1.0\textwidth}

NAME: Alan Scott

Student ID: 400263658

\noindent
\rule[3 mm]{\textwidth}{0.5mm}

This examination paper includes \noofpages pages and
8 % VARIABILITY
questions. You are responsible for ensuring that your copy of the examination
paper is complete. Bring any discrepancy to the attention
of your instructor.\\

\noindent
\emph{By submitting this work, I certify that the work represents solely my own
independent efforts. I confirm that I am expected to exhibit honesty and use
ethical behaviour in all aspects of the learning process.  I confirm that it is
my responsibility to understand what constitutes academic dishonesty under the
\href{https://secretariat.mcmaster.ca/app/uploads/Academic-Integrity-Policy-1-1.pdf}
{Academic Integrity Policy}}.\\

\noindent
\textbf{Special Instructions}:

\begin{enumerate}

\item For taking tests remotely: 
\begin{itemize}
\item Turn off all unnecessary programs, especially Netflix, YouTube, games like
  Xbox or PS4, anything that might be downloading or streaming.
\item If your house is shared, ask others to refrain from doing those activities
  during the test.
\item If you can, connect to the internet via a wired connection.
\item Move close to the Wi-Fi hub in your house. 
\item Restart your computer, 1-2 hours before the exam. A restart can be very
  helpful for several computer hiccups.
\item Use a VPN (Virtual Private Network) since this improves the connection to
  the CAS servers.
\item Commit and push your tex file, compiled pdf file, and code files
  frequently.  As a minimum you should do a commit and push after completing
  each question.
\item Ensure that you push your solution (tex file, pdf file and code files)
  before time expires on the test.  The solution that is in the repo at the
  deadline is the solution that will be graded.
\item If you have trouble with your git repo, the quickest solution may be to
  create a fresh clone.
\end{itemize}
\item It is your responsibility to ensure that the answer sheet is properly
  completed. Your examination result depends upon proper attention to the
  instructions.
\item All physical external resources are permitted, including textbooks, calculators,
  computers, compilers, and the internet.
\item The work has to be completed individually.  Discussion with others is
  strictly prohibited.
\item Read each question carefully.
\item Try to allocate your time sensibly and divide it appropriately between the
  questions.  Use the allocated marks as a guide on how to divide your time
  between questions.
\item The quality of written answers will be considered during grading.  Please
  make your answers well-written and succinct.
\item The set $\mathbb{N}$ is assumed to include $0$.
\end{enumerate}
%\end{minipage}\\

\examheader{CS2ME3/SE2AA4 \ifthenelse{\equal{\soln}{y}} {\hfill SOLUTIONS} }

\renewcommand{\labelenumi}{\Alph{enumi}.}

\newpage

%%%%%%%%%%%%%%%%%%%%%%%%%%%%%%%%%%%%%%%%%%%%%%%%%%%%%%%%%%%%%%%%%%%%%%

\question{5 marks} What are the problems with using ``average lines of code
written per day'' as a metric for programmer productivity?

\bigskip

\noindent \wss{Provide your reasons in the itemized list below.  Add more items
  as required.}

\begin{itemize}
\item ALOC/day becomes innacurate for reusable code, as this code can often be copy-pasted, which is minimal effort on the coder's part, but produces the same number of lines.
\item ALOC/day encourages the overabundance of code, which can usually lead to unnecessary and/or difficult to understand code.
\item ALOC/day can vary depending on the program or the language (ie Java programs will normally lead to more lines than Python code due to the differnce in method inclusion, since Java programs use an extra line for a '\{' character.
\item ALOC/day does not take into account the actual content of the code, as coding a more complex algorithm may take fewer lines than coding something basic (for example a 30 line algorithm will likely take more time than 50 lines of System.out.println()).
\end{itemize}

%%%%%%%%%%%%%%%%%%%%%%%%%%%%%%%%%%%%%%%%%%%%%%%%%%%%%%%%%%%%%%%%%%%%%%

\newpage

\question{5 marks} Critique the following requirements specification
for a new cell phone application, called CellApp.  Use the following criteria
discussed in class for judging the quality of the specification: abstract,
unambiguous, and validatable.  How could you improve the requirements
specification?

\bigskip

``The user shall find CellApp easy to use.''

\bigskip

\noindent \wss{Fill in the itemized list below with your answers.  Leave the
  word in bold at the beginning of each item.}

\begin{itemize}
\item \textbf{Abstract} - The specification specifies a given app, rather than referring to an app abstractly.
\item \textbf{Unambiguous} - The specification is very ambiguous, as the term "easy to use" is not defined.
\item \textbf{Validatable} - The specification is not validatable, as there is no given metric with which ease of use is to be measured.
\item \textbf{How to improve} - The specification could be improved by further defining the term "easy to use" and giving a way for it to be measured.
\end{itemize}

%%%%%%%%%%%%%%%%%%%%%%%%%%%%%%%%%%

\newpage

\question{5 marks} The following module is proposed for the maze tracing robot
we discussed in class (L20).  This module is a leaf module in the decomposition
by secrets hierarchy.

\begin{description}
\item [Module Name] find\_path
\item [Module Secret] The data structure and algorithm for finding the shortest
  path in a graph.
\end{description}

\noindent \wss{Fill in the answers to the questions below.  For each item you
  should leave the bold question and write your answer directly after it.}

\begin{enumerate}
\item \textbf{Is this module Hardware Hiding, Software Decision Hiding or
    Behaviour Hiding?  Why?}

 The given module is Software Decision Hiding, since it hides an algorithm and data structure within a software module.

\item \textbf{Is this a good secret?  Why?}

This is not a good secret as it refers to two separate things: a data structure and an algorithm.

\item \textbf{Does the specification for maze tracing robot require environment
    variables?  If so, which environment variables are needed?}

The robot would need some environment variables representing the current position of the robot, as well as the layout of the map (along with the start and end points).

\end{enumerate}

%%%%%%%%%%%%%%%%%%%%%%%%%%%%%%%%%%

\newpage

\question{5 marks} Answer the following questions assuming that you are in doing
your final year capstone in a group of 5 students.  Your project is to write a
video game for playing chess, either over the network between two human
opponents, or locally between a human and an Artificial Intelligence (AI)
opponent.

\bigskip

\noindent \wss{Fill in the answers to the questions below.  For each item you
  should leave the bold question and write your answer directly after it.}

\begin{enumerate}
  
\item \textbf{You have 8 months to work on the project.  Keeping in mind that we
  usually need to fake a rational design process, what major milestones and what
  timeline for achieving these milestones do you propose?  You can indicate the
  time a milestone is reached by the number of months from the project's start date.}

The milestones can be modelled after the stages of the design process, with the time assigned to each relative to the time it would take to undertake them. The first half-month would be dedicated to determining what problems would be solved, and what requirements would be met. This milestone would be met with the completion of the SRS. The next month would be dedicated to coming up with the design for the program, and would reach its milestone with the completion of the MIS document. The largest portion of the time would come from the implementation phase, which would take place over a period of three months following the completion of the design stage. The milestone for this phase would be reached with a running implementation of the game (ie the code), whether or not it functions optimally. The following three months would be assigned to verification, involving a full verification and testing of the code. This milestone is complete when a full VnV report is produced. The last half month is allocated for delivery and maintenance, a milestone which is automatically met at the end of the 8 month project time.
  
\item \textbf{Everything in your process should be verified, including the
    verification.  How might you verify your verification?}

The process of verification can be verified with a nV report, which will verify that our verification is indeed verifying correct behaviour.
  
\item \textbf{How do you propose verifying the installability of your game?}

The installability of the game can be verified by attempting to install it on multiple different hardwares and operating systems to verify that it works on a variety of platforms. The program should also be tested by several people, in order to verfity that the install process is intuitive and easy to use.
  
\end{enumerate}

%%%%%%%%%%%%%%%%%%%%%%%%%%%%%%%%%%%%%%%%%%%%%%%%%%%%%%%%%%%%%%%%%%%%%%

\newpage

\question{5 marks} As for the previous question, assume you are doing a final
year capstone project in a group of 5 students.  As above, your project
is to write a video game for playing chess, either over the network between two
human opponents, or locally between a human and an Artificial Intelligence
(AI) opponent.  The questions below focus on verification and testing.

\bigskip

\noindent \wss{Fill in the answers to the questions below.  For each item you
  should leave the bold question and right your answer directly after it.}

\begin{enumerate}
\item \textbf{Assume you have 4 work weeks (a work week is 5 days) over the
    course of the project for verification activities.  How many collective
    hours do you estimate that your team has available for verification related
    activities?  Please justify your answer.}

Given that there are 5 students per group, 4 work weeks and 5 work days per week, we can say that the the collective time for the group is 80 times the average student's daily available work time. Considering that a work day implies the student has other classes (or even work) happening concurrently, we can say that the student does not have the full day to work on the project. We can assume that the average student has about 2-3 hours that they are able and willing to work on this project per week, bring the total collective time to a lowball estimate of approximately 160 hours total for testing. 
  
\item \textbf{Given the estimated hours available for verification, what verification
    techniques do you recommend for your team?  Please list the techniques,
    along with the number of hours your team will spend on each technique, and
    the reason for selecting this technique.}

For various parts of the program, both dynamic and static testing should be performed. For the player vs player portion of the game, not much testing will need to be performed, as the program does not need to make any non-obvious computations (only needing to check for things like checks and captures, and not lines of play). This testing can be performed manually by two people, and is to be done dynamically (since they will need to be playing the game). It is possible for this process to be automated, but there isnt much point as automation would take an unnecessary extra amount of time. Since this part requires less time to test, it will only require about 40 hours of testing (this part will likely be worked on by everyone since the AI portion will require that this part work correctly). As for the AI portion, it will take significantly longer, as it is far more complicated. This testing will likely be performed dynamically and automatically, via a form of unit test. Essentially, a board position will be entered, and the AI's next move will be compared against another chess engine such as Stockfish, and then rated for accuracy. This process can be automated to save time. This technique, along with the process of automating, would take 60 hours. Once case by case testing is performed, a more long form dynamic test can be performed by replacing the human player with another chess engine, and analyzing the games played, and specifically finding any faults. This process could also be automated to save time. This stage would also take about 60 hours.
  
\item \textbf{Is the oracle problem a concern for implementing your game?  Why
    or why not?  If it is a concern, how do you recommend testing your software?}

The oracle problem is a concern for the game, specifically for the AI portion of the game. Since the AI is meant to emulate a human player, it is very difficult to decide exactly what actions the AI should take, especially since the best moves in chess are not always immediately clear to an average person. To test the software with this concern, we could use prevous GM level games of chess, and see how closely the AI determines its moves relative to the real people, identifying any faults as we do so.
    
\end{enumerate}

%%%%%%%%%%%%%%%%%%%%%%%%%%%%%%%%%%%%%%%%%%%%%%%%%%%%%%%%%%%%%%%%%%%%%%

\newpage

\question{5 marks} Consider the following natural language specification for a
function that looks for resonance when the input matches an integer multiple of
the wavelengths 5 and 7. Provided an integer input between 1 and 1000, the
function returns a string as specified below:

\begin{itemize}
\item If the number is a multiple of 5, then the output is “resonance 5”
\item If the number is a multiple of 7, then the output is “resonance 7”
\item If the number is a multiple of both 5 and 7, then the output is “resonance
  5 and 7”
\item Otherwise, the output is “no resonance”
\end{itemize}

You can assume that inputs outside of the range 1 to 1000 do not occur.

\begin{enumerate}
\item What are the sets $D_i$ that partition $D$ (the input domain) into a
  reasonable set of equivalence classes? \\
The sets that partition D are numbers are: $x \in \{x / 7 \land \lnot x / 5\}$, $x \in \{x / 5 \land \lnot x / 7\}$, $x \in \{\lnot x / 5 \land \lnot x / 7\}$, and $x \in \{x / 5 \land x / 7\}$, where '/' represents integer divisibility.


\item Given the sets $D_i$, and the heuristics discussed in class, how would you
  go about selecting test cases? \\

The test cases would be select that there would be at least one input from each of the sets $D_i$, as this would cover the full range of normal operation. In addition, additional test cases would be conducted for inputs at and close to 1 and 1000, in order to test the expected performance at the bounds. Since the function assumes that the numbers are within [1,1000], no other numbers need be tested. 
  
\end{enumerate}
  
%%%%%%%%%%%%%%%%%%%%%%%%%%%%%%%%%%

\newpage

\question{5 marks} Below is a partial specification for an MIS for the game of
tic-tac-toe (\url{https://en.wikipedia.org/wiki/Tic-tac-toe}).  You should
complete the specification.

\bigskip

\wss{The parts that you need to fill in are marked by comments, like this one.
  You can use the given local functions to complete the missing specifications.
  You should not have to add any new local functions, but you can if you feel it
  is necessary for your solution.  As you edit the tex source, please leave the
  \texttt{wss} comments in the file.  You can put your answer immediately
  following the comment.}

\subsection* {Syntax}

\subsubsection* {Exported Constants}

SIZE = 3 {\it //size of the board in each direction}\\

\subsubsection* {Exported Types}

cellT = \{ X, O, FREE \} \\

\subsubsection* {Exported Access Programs}

\begin{tabular}{| l | l | l | p{7cm} |}
\hline
\textbf{Routine name} & \textbf{In} & \textbf{Out} & \textbf{Exceptions}\\
\hline
init & ~ & ~ & ~\\
\hline
move & $\mathbb{N}$, $\mathbb{N}$ & ~ & OutOfBoundsException, InvalidMoveException\\
\hline
getb & $\mathbb{N}$, $\mathbb{N}$ & cellT & OutOfBoundsException\\
\hline
get\_turn & ~ & cellT & ~\\
\hline
is\_valid\_move & $\mathbb{N}$, $\mathbb{N}$ & $\mathbb{B}$ & OutOfBoundsException\\
\hline
is\_winner & cellT & $\mathbb{B}$ & ~\\
\hline
is\_game\_over & ~ & $\mathbb{B}$ & ~\\
\hline

\end{tabular}

\subsection* {Semantics}

\subsubsection* {State Variables}

$b$: boardT\\
$\mathit{Xturn}$: $\mathbb{B}$

\subsubsection* {State Invariant}

\wss{Place your state invariant or invariants here} None

\subsubsection* {Assumptions}

The init method is called for the abstract object before any other access routine is called for that
object.  The init method can be used to return the state of the game to the state of a new game.

\subsubsection* {Access Routine Semantics}

init():
\begin{itemize}
\item transition: 
$$\mathit{Xturn}, b := \text{true}, 
< \begin{array}{c}
< \mbox{FREE}, \mbox{FREE}, \mbox{FREE} >\\
< \mbox{FREE}, \mbox{FREE}, \mbox{FREE} >\\
< \mbox{FREE}, \mbox{FREE}, \mbox{FREE} >\\
\end{array} >
$$
\item exception: none
\end{itemize}

\noindent move($i$, $j$):
\begin{itemize}
\item transition: $\mathit{Xturn}, b[i, j] := \neg \mathit{Xturn}, (\mathit{Xturn} \Rightarrow \mbox{X} | \neg
\mathit{Xturn} \Rightarrow \mbox{O})$
\item exception
$$exc := (\mbox{InvalidPosition}(i, j) \Rightarrow \mbox{OutOfBoundsException} | \neg \mbox{is\_valid\_move}(i, j)
\Rightarrow \mbox{InvalidMoveException})$$
\end{itemize}

\noindent getb(i, j):
\begin{itemize}
\item output: $\mathit{out} := b[i, j]$
\item exception
$exc := (\mbox{InvalidPosition}(i, j) \Rightarrow \mbox{OutOfBoundsException})$
\end{itemize}

\noindent get\_turn():
\begin{itemize}
\item output: \wss{Return the cellT that corresponds to the current turn}\\
$out := (XTurn \implies X | True \implies O)$

\item exception: none
\end{itemize}

\noindent is\_valid\_move(i, j):
\begin{itemize}
\item output: $\mathit{out} := (b[i][j] = \mbox{FREE})$ 
\item exception $exc := (\mbox{InvalidPosition}(i, j) \Rightarrow \mbox{OutOfBoundsException})$
\end{itemize}

\noindent is\_winner(c):
\begin{itemize}
\item output: $\mathit{out} := \mbox{horizontal\_win}(c, b) \vee \mbox{vertical\_win}(c, b) \vee
\mbox{diagonal\_win}(c, b)$ 
\item exception: none
\end{itemize}

\noindent is\_game\_over():
\begin{itemize}
\item output: \wss{Returns true if X or O wins, or if there are no more moves remaining}\\
$out := (isWinner(X) \lor isWinner(O)\implies True) | \forall(x : cellT | x \in b : x \neq \mbox{FREE}) \implies True | True \implies False$

\item exception: none
\end{itemize}

\subsubsection* {Local Types}

boardT = sequence [SIZE, SIZE] of cellT

\subsubsection* {Local Functions}

\noindent \textbf{InvalidPosition}: $\mathbb{N}$ $\times$ $\mathbb{N}$ $\rightarrow$ $\mathbb{B}$\\
~\newline
InvalidPosition$(i, j) \equiv \neg ( ( 0 \leq i < \mbox{SIZE} ) \wedge ( 0 \leq j < \mbox{SIZE}))$

~\newline

\noindent \textbf{count}: cellT $\rightarrow$ $\mathbb{N}$\\
~\newline
\wss{For the current board return the number of occurrences of the cellT
  argument}
~\newline
count$(c) \equiv +(x : cellT | x \in b \land x = c : 1)$

~\newline

\noindent \textbf{horizontal\_win} : cellT $\times$ boardT $\rightarrow$ $\mathbb{B}$\\
~\newline
horizontal\_win$(c, b) \equiv \exists (i : \mathbb{N} | 0 \leq i < \mbox{SIZE} : b[i, 0] = b[i, 1] = b[i, 2] = c)$

~\newline

\noindent \textbf{vertical\_win} : cellT $\times$ boardT $\rightarrow$ $\mathbb{B}$\\
~\newline
vertical\_win$(c, b) \equiv \exists (j : \mathbb{N} | 0 \leq j < \mbox{SIZE} : b[0, j] = b[1, j] = b[2, j] = c)$

~\newline

\noindent \textbf{diagonal\_win} : cellT $\times$ boardT $\rightarrow$ $\mathbb{B}$\\
~\newline
\wss{Returns true if one of the diagonals for the board has all of the entries
  equal to cellT} 
~\newline
diagonal\_win$(c, b) \equiv (\forall(i : \mathbb{N}|0 \leq i < \mbox{SIZE}: b[i,i]=c)) \lor (b[2,0]=b[1,1]=b[0,2]=c)$


%%%%%%%%%%%%%%%%%%%%%%%%%%%%%%%%%%

\newpage

\question{5 marks} For this question you will implement in Java an ADT for a 1D
sequence of real numbers.  We want to take the mean of the numbers in the
sequence, but as the following web-page shows, there are several different
algorithms for doing this: \url{https://en.wikipedia.org/wiki/Generalized_mean}

Given that there are different options, we will use the strategy design pattern,
as illustrated in the following UML diagram:

\begin{figure}[!h]
\begin{center}
\includegraphics[scale=0.7]{Seq1D_Mean_Strategy_UML.png}
\end{center}
\caption{UML Class Diagram for Seq1D with Mean Function, using Strategy
  Pattern} \label{Fig_UML_Strategy}
\end{figure}

You will need to fill in the following blank files:
\texttt{MeanCalculator.java}, \texttt{HarmonicMean.java},
\texttt{QuadraticMean.java}, and \texttt{Seq1D.java}.  Two testing files are
also provided: \texttt{Expt.java} and \texttt{TestSeq1D.java}.  The file
\texttt{Expt.java} is pre-populated with some simple experiments to help you see
the interface in use, and do some initial testing.  You are free to add to this
file to experiment with your work, but the file itself isn't graded.  The
\texttt{TestSeq1D.java} is also not graded.  However, you may want to create
test cases to improve your confidence in your solution.  The stubs of the
necessary files are already available in your \texttt{src} folder.  The code
will automatically be imported into this document when the \texttt{tex} file is
compiled.  You should use the provided Makefile to test your code.  You will NOT
need to modify the Makefile.  The given Makefile will work for \texttt{make
  test}, without errors, from the initial state of your repo.  The \texttt{make
  expt} rule will also work, because all lines of code have been commented out.
Uncomment lines as you complete work on each part of the modules relevant to
those lines in \texttt{Expt.java} file.  As usual, the final test is whether the
code runs on mills.  You do not need to worry about doxygen comments.

Any exceptions in the specification have names identical to the expected Java
exceptions; your code should use exactly the exception names as given in the
spec.

Remember, your code needs to implement the given specification so that the
interface behaves as specified.  This does NOT mean that the local functions
need to all be implemented, or that the types used internally to the spec need
to be implemented exactly as given.  If you do implement any local functions,
please make them private.  The real type in the MIS should be implemented by
\texttt{Double} (capital D) in Java.

\wss{Complete Java code to match the following specification.}

%%%%%%%%%%%%%%%%%%%%%%%%%%%%%%%%%%

\newpage

\section* {Mean Calculator Interface Module}

\subsection*{Interface Module}

MeanCalculator

\subsection* {Uses}

None

\subsection* {Syntax}

\subsubsection* {Exported Constants}

None

\subsubsection* {Exported Types}

None 

\subsubsection* {Exported Access Programs}

\begin{tabular}{| l | l | l | p{5cm} |}
\hline
\textbf{Routine name} & \textbf{In} & \textbf{Out} & \textbf{Exceptions}\\
\hline
meanCalc & seq of $\mathbb{R}$ & $\mathbb{R}$ & ~\\
\hline
\end{tabular}

\subsubsection* {Considerations}

meanCalc calculates the mean (a real value) from a given sequence of reals.
The order of the entries in the sequence does not matter.

%%%%%%%%%%%%%%%%%%%%%%%%%%%%%%%%%%

\newpage

\section* {Harmonic Mean Calculation}

\subsection*{Template Module inherits MeanCalculator}

HarmonicMean

\subsection* {Uses}

MeanCalculator

\subsection* {Syntax}

\subsubsection* {Exported Constants}

None

\subsubsection* {Exported Types}

None 

\subsubsection* {Exported Access Programs}

\begin{tabular}{| l | l | l | p{5cm} |}
\hline
\textbf{Routine name} & \textbf{In} & \textbf{Out} & \textbf{Exceptions}\\
\hline
meanCalc & seq of $\mathbb{R}$ & $\mathbb{R}$ & ~\\
\hline
\end{tabular}

\subsection* {Semantics}

\subsubsection* {State Variables}

None

\subsubsection* {State Invariant}

None

\subsubsection* {Assumptions}

None

\subsubsection* {Access Routine Semantics}

meanCalc($v$)
\begin{itemize}
\item output: $\mathit{out} := \frac{|x|}{+(x: \mathbb{R} | x \in v : 1/x)}$
\item exception: none
\end{itemize}

%%%%%%%%%%%%%%%%%%%%%%%%%%%%%%%%%%

\newpage

\section* {Quadratic Mean Calculation}

\subsection*{Template Module inherits MeanCalculator}

QuadraticMean

\subsection* {Uses}

MeanCalculator

\subsection* {Syntax}

\subsubsection* {Exported Constants}

None

\subsubsection* {Exported Types}

None 

\subsubsection* {Exported Access Programs}

\begin{tabular}{| l | l | l | p{5cm} |}
\hline
\textbf{Routine name} & \textbf{In} & \textbf{Out} & \textbf{Exceptions}\\
\hline
meanCalc & seq of $\mathbb{R}$ & $\mathbb{R}$ & ~\\
\hline
\end{tabular}

\subsection* {Semantics}

\subsubsection* {State Variables}

None

\subsubsection* {State Invariant}

None

\subsubsection* {Assumptions}

None

\subsubsection* {Access Routine Semantics}

meanCalc($v$)
\begin{itemize}
\item output: $\mathit{out} := \sqrt{\frac{+(x: \mathbb{R} | x \in v : x^2)}{|x|}}$
\item exception: none
\end{itemize}

%%%%%%%%%%%%%%%%%%%%%%%%%%%%%%%%%%

\newpage

\section* {Seq1D Module}

\subsection* {Template Module}

Seq1D

\subsection* {Uses}

MeanCalculator

\subsection* {Syntax}

\subsubsection* {Exported Types}

Seq1D = ?

\subsubsection* {Exported Constants}

None

\subsubsection* {Exported Access Programs}

\begin{tabular}{| l | l | l | p{6cm} |}
\hline
\textbf{Routine name} & \textbf{In} & \textbf{Out} & \textbf{Exceptions}\\
\hline
new Seq1D & seq of $\mathbb{R}$, MeanCalculator & Seq1D & IllegalArgumentException\\
\hline
setMaxCalculator & MaxCalculator &  & \\
\hline
mean &  & $\mathbb{R}$ & \\
\hline

\end{tabular}

\subsection* {Semantics}

\subsubsection* {State Variables}

$s$: seq of $\mathbb{R}$\\
meanCalculator: MeanCalculator

\subsubsection* {State Invariant}

None

\subsubsection* {Assumptions}

\begin{itemize}
\item The Seq1D constructor is called for each object instance before any other
  access routine is called for that object.  The constructor can only be called
  once.  All real numbers provided to the constructor will be zero or positive.
\end{itemize}

\subsubsection* {Access Routine Semantics}

new Seq1D($x$, $m$):
\begin{itemize}
\item transition: $s, \text{meanCalculator} := x, m$
\item output: $\mathit{out} := \mathit{self}$
\item exception:
  $\mathit{exc} := (|x| = 0 \Rightarrow \mbox{IllegalArgumentException})$
\end{itemize}

\noindent setMeanCalculator($m$):
\begin{itemize}
\item transition: $\mbox{meanCalculator} := m$
\item exception: none
\end{itemize}

\noindent mean():
\begin{itemize}
\item output: $\mathit{out} := \mbox{meanCalculator.meanCalc}()$
\item exception: none
\end{itemize}

%%%%%%%%%%%%%%%%%%%%%%%%%%%%%%%%%%

\newpage

\subsection*{Code for MeanCalculator.java}

\noindent \lstinputlisting[language = Java]{./src/MeanCalculator.java}

\newpage

\subsection*{Code for HarmonicMean.java}

\noindent \lstinputlisting[language = Java]{./src/HarmonicMean.java}

\newpage

\subsection*{Code for QuadraticMean.java}

\noindent \lstinputlisting[language = Java]{./src/QuadraticMean.java}

\newpage

\subsection*{Code for Seq1D.java}

\noindent \lstinputlisting[language = Java]{./src/Seq1D.java}

%%%%%%%%%%%%%%%%%%%%%%%%%%%%%%%%%%

\end{document}